\chapter{Landasan Teori}
\label{chap:LandasanTeori}

\section{Play Framework}
\label{sec:play}
Play \textit{Framework} merupakan sebuah \textit{framework} untuk membuat \textit{web} dengan menggunakan bahasa Java dan Scala. Play \textit{Framework} menggunakan konsep MVC (Model, View, dan Controller). 

\section{Openlayers}
\label{sec:openlayers}
Openlayers adalah \textit{library} untuk menampilkan peta pada \textit{web browser} terkini dengan menggunakan bahasa Javascript. Openlayers mempunyai performa yang baik untuk ditampilkan pada \textit{web browser}. Modul yang dipakai dalam Openlayers adalah:
\begin{enumerate}
	\item 	Bing Maps untuk menampilkan peta menggunakan Bing.
	\item Draw untuk menggambar poin pada peta.
\end{enumerate}

\section{Zurb Foundation}
\label{sec:foundation}
Zurb Foundation adalah sebuah \textit{framework} untuk membuat tampilan \textit{web} menjadi responsif. Karena tampilan KIRI menggunakan Javascript, maka Zurb Foundation yang digunakan adalah Javascript. Modul yang dipakai dalam Zurb Foundation adalah:
\begin{enumerate}
	\item jquery.js dan fastclick.js untuk instalasi Zurb Foundation.
	\item foundation.min.js untuk memuat fungsi utama Zurb Foundation dan semua \textit{plugin} Javascript.
	\item foundation.alert.js untuk menampilkan \textit{alert} pada \textit{web}.
\end{enumerate}




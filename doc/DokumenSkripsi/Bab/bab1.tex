\chapter{Pendahuluan}
\label{chap:pendahuluan}

\section{Latar Belakang}
\label{sec:latarbelakang}

KIRI (http://kiri.travel) merupakan aplikasi website yang membantu pengguna bepergian baik dalam kota maupun luar kota. Jika dalam kota, KIRI akan menentukan angkutan kota yang tersedia di kota tersebut, jika luar kota, maka KIRI menentukan travel yang tersedia ke kota yang akan dituju serta angkutan kota menuju tempat tujuan. Saat ini, KIRI tersedia dalam berbagai kota, yaitu Bandung, Depok, Jakarta, Surabaya, dan Malang. KIRI menyediakan berbagai rute alternatif yang dapat dipilih oleh pengguna. KIRI juga dapat membimbing pengguna langkah demi langkah untuk mencapai lokasi tujuan. 

Dalam pengembangan website, kita sering menjumpai bahasa yang dipakai adalah bahasa PHP. PHP tersebut kurang cocok dengan proyek besar. Masalah yang sering dijumpai seperti tidak ada deklarasi variabel, tidak ada tipe variabel. Dalam pengembangan website terdapat berbagai macam framework. Framework adalah kerangka yang membantu pengguna untuk menyelesaikan website. 

Dari berbagai framework yang dapat digunakan, dipilih Play Framework. Play Framework merupakan framework untuk membuat website dengan bahasa pemrograman Java. Play Framework menerapkan konsep MVC, yaitu Model, View, dan Controller. Dalam penelitian ini, Play Framework dipakai karena Play Framework terstruktur dan umum. 

\section{Rumusan Masalah}
\label{rumusanMasalah}
\begin{itemize}
	\item Bagaimana memahami dan menganalisa kode KIRI yang sudah ada?
	\item Bagaimana melakukan porting kode KIRI (PHP) menjadi Play Framework (Java) ?
\end{itemize}

\section{Tujuan}
\label{sec:tujuan}
\begin{itemize}
	\item Memahami dan menganalisa kode KIRI.
	\item Menjadikan kode KIRI menjadi Play Framework.
\end{itemize}

\section{Batasan Masalah}
\label{sec:batasanMasalah}
\begin{enumerate}
	\item Lorem ipsum
\end{enumerate}

\section{Metode Penelitian}
\label{sec:metodePenelitian}
Berikut adalah metode penelitian yang digunakan dalam pembuatan skripsi ini:
	\begin{enumerate}
		\item Memahami dan melakukan analisa kode KIRI yang sudah ada.
		\item Melakukan studi literatur tentang metode yang berkaitan dengan kode PHP dan Java (Play Framework).
		\item Merancang dan mengimplementasikan kode KIRI yang sudah ada menjadi Play Framework.
		\item Melakukan pengujian dan eksperimen.
		\item Membuat dokumen skripsi.
	\end{enumerate}
	
\section{Sistematika Penulisan}
\label{sec:sistematikaPenulisan}
Setiap bab dalam penulisan ini memiliki sistematika yang dijelaskan ke dalam poin-poin sebagai berikut:
	\begin{enumerate}
		\item Bab 1: Pendahuluan, yaitu membahas tentang latar belakang, rumusan masalah, tujuan, batasan masalah, metode penelitian dan sistematika penulisan.
		\item Bab 2: Dasar Teori, yaitu membahas mengenai teori-teori yang mendukung berjalannya skripsi ini yang berisi tentang penggunaan Play Framework.
		\item Bab 3: Analisis, yaitu membahas mengenai analisis masalah yang berisi tentang kode KIRI \textit{Front-End Server Side} serta melakukan \textit{porting} kode KIRI \textit{Front-End Server Side} menjadi Play Framework.
	\end{enumerate}
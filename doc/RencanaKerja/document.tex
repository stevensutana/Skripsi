\documentclass[a4paper,twoside]{article}
\usepackage[T1]{fontenc}
\usepackage[bahasa]{babel}
\usepackage{graphicx}
\usepackage{graphics}
\usepackage{float}
\usepackage[cm]{fullpage}
\pagestyle{myheadings}
\usepackage{etoolbox}
\usepackage{setspace} 
\usepackage{lipsum} 
\setlength{\headsep}{30pt}
\usepackage[inner=2cm,outer=2.5cm,top=2.5cm,bottom=2cm]{geometry} %margin
% \pagestyle{empty}

\makeatletter
\renewcommand{\@maketitle} {\begin{center} {\LARGE \textbf{ \textsc{\@title}} \par} \bigskip {\large \textbf{\textsc{\@author}} }\end{center} }
\renewcommand{\thispagestyle}[1]{}
\markright{\textbf{\textsc{AIF401 \textemdash Rencana Kerja Skripsi \textemdash Sem. Ganjil 2015/2016}}}

\onehalfspacing
 
\begin{document}

\title{\@judultopik}
\author{\nama \textendash \@npm} 

%tulis nama dan NPM anda di sini:
\newcommand{\nama}{Steven Sutana}
\newcommand{\@npm}{2012730046}
\newcommand{\@judultopik}{Porting PHP menjadi Play Framework (KIRI \textit{Front-End})} % Judul/topik anda
\newcommand{\jumpemb}{1} % Jumlah pembimbing, 1 atau 2
\newcommand{\tanggal}{26/08/2015}
\maketitle

\pagenumbering{arabic}

\section{Deskripsi}
KIRI merupakan aplikasi yang membantu pengguna bepergian baik dalam kota maupun luar kota. Jika dalam kota, KIRI akan menentukan angkutan kota yang tersedia di kota tersebut, jika luar kota, maka KIRI menentukan travel yang tersedia ke kota yang akan dituju. KIRI Travel tersedia dalam berbagai kota, yaitu Bandung, Depok, Jakarta, Surabaya, dan Malang. KIRI menyediakan berbagai rute alternatif yang dapat dipilih oleh pengguna. KIRI juga dapat membimbing pengguna langkah demi langkah untuk mencapai lokasi tujuan. 

Dalam pengembangan website, kita sering menjumpai bahasa yang dipakai adalah bahasa PHP. PHP tersebut kurang cocok dengan proyek besar. Masalah yang sering dijumpai seperti tidak ada deklarasi variabel, tidak ada tipe variabel. PHP merupakan bahasa yang asing bagi mahasiswa/i IT UNPAR karena bahasa pemrograman yang dipakai adalah bahasa Java. Java sudah menjadi bahasa umum yang dikenal oleh mahasiswa IT UNPAR. Play Framework merupakan framework untuk membuat website dengan bahasa pemrograman Java. Dengan melakukan penelitian ini, mahasiswa dilatih untuk memahami dan menganalisa kode KIRI menjadi Play Framework.

\section{Rumusan Masalah}
\begin{itemize}
	\item Bagaimana memahami dan menganalisa kode KIRI yang sudah ada?
	\item Bagaimana melakukan porting kode KIRI (PHP) menjadi Play Framework (Java) ?
\end{itemize}

\section{Tujuan}
\begin{itemize}
	\item Memahami dan menganalisa kode KIRI.
	\item Menjadikan kode KIRI menjadi Play Framework yang lebih terstruktur dan umum.
\end{itemize}

\section{Deskripsi Perangkat Lunak}
Perangkat lunak akhir yang akan dibuat memiliki fitur minimal sebagai berikut:
\begin{itemize}
	\item Perangkat lunak dapat menampilkan tampilan KIRI secara terstruktur dengan menggunakan Play Framework.
	\item Perangkat lunak dapat berfungsi seperti website KIRI yang sudah ada sebagai \textit{Front-End Server Side}.
\end{itemize}

\section{Rencana Kerja}

Rencana kerja untuk menyelesaikan skripsi ini:
\begin{itemize}
	\item Pada saat mengambil kuliah AIF401 Skripsi 1
	\begin{enumerate}
		\item Memahami dan melakukan analisa kode KIRI yang sudah ada.
		\item Melakukan studi literatur tentang metode yang berkaitan dengan porting PHP menjadi Java (Play Framework).
		\item Mempelajari fitur-fitur Play Framework untuk membuat website yang terstrtuktur.
		\item Merancang langkah-langkah untuk membuat website menggunakan Play Framework
	\end{enumerate}
	\item Pada saat mengambil kuliah AIF401 Skripsi 2
	\begin{enumerate}
		\item Merancang dan mengimplementasikan kode KIRI yang sudah ada menjadi Play Framework.
		\item Melakukan pengujian dan eksperimen.
		\item Membuat dokumentasi skripsi.
	\end{enumerate}
\end{itemize}

\section{Isi {\it Progress Report} Skripsi 1}
Isi dari {\it Progress Report} Skripsi 1 yang akan diselesaikan dan dilaporkan ke pembimbing paling lambat 2 minggu sebelum tenggat waktu yang ditetapkan koordinator adalah :
\begin{enumerate}
	\item Mengerti kode KIRI yang sudah ada.
	\item Mengerti dan menyelesaikan dokumentasi Dasar Teori dan Analisis.
\end{enumerate}
Estimasi persentase penyelesaian skripsi sampai dengan {\it Progress Report} Skripsi 1 adalah : 40\%

\section{Pernyataan Khusus}
Berlatar belakang perihal kejujuran serta keterbasan jumlah dosen, saya menyatakan akan mematuhi aturan-aturan khusus berikut:
\begin{enumerate}
	\item Skripsi adalah hasil karya saya sendiri. Peran teman / orang lain adalah untuk membantu pemahaman, tetapi tidak dalam konten Skripsi.
	\item Saya menetapkan batasan yang jelas antara konten saya, dengan buatan orang lain (termasuk kode yang diambil dari {\it open source project})
	\item Pengambilan kedua hanya akan dilakukan hanya jika sudah memenuhi minimal 90\% dari target.
\end{enumerate}
Saya bersedia mematuhi peraturan di atas, dan bersedia menerima sanksi pembatalan pengambilan Skripsi dengan dosen pembimbing terkait jika terbukti melanggar. Peraturan ini berlaku pada Skripsi 1 dan 2.

\vspace{1.5cm}

\centering Bandung, \tanggal\\
\vspace{2cm} \nama \\ 
\vspace{1cm}

Menyetujui, \\
\ifdefstring{\jumpemb}{2}{
\vspace{1.5cm}
\begin{centering} Menyetujui,\\ \end{centering} \vspace{0.75cm}
\begin{minipage}[b]{0.45\linewidth}
% \centering Bandung, \makebox[0.5cm]{\hrulefill}/\makebox[0.5cm]{\hrulefill}/2013 \\
\vspace{2cm} Nama: \makebox[3cm]{\hrulefill}\\ Pembimbing Utama
\end{minipage} \hspace{0.5cm}
\begin{minipage}[b]{0.45\linewidth}
% \centering Bandung, \makebox[0.5cm]{\hrulefill}/\makebox[0.5cm]{\hrulefill}/2013\\
\vspace{2cm} Nama: \makebox[3cm]{\hrulefill}\\ Pembimbing Pendamping
\end{minipage}
\vspace{0.5cm}
}{
% \centering Bandung, \makebox[0.5cm]{\hrulefill}/\makebox[0.5cm]{\hrulefill}/2013\\
\vspace{2cm} Nama: \makebox[3cm]{\hrulefill}\\ Pembimbing Tunggal
}

\end{document}


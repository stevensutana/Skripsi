\documentclass[a4paper,twoside]{article}
\usepackage[T1]{fontenc}
\usepackage[bahasa]{babel}
\usepackage{graphicx}
\usepackage{graphics}
\usepackage{float}
\usepackage[cm]{fullpage}
\pagestyle{myheadings}
\usepackage{etoolbox}
\usepackage{setspace} 
\usepackage{lipsum} 
\setlength{\headsep}{30pt}
\usepackage[inner=2cm,outer=2.5cm,top=2.5cm,bottom=2cm]{geometry} %margin
% \pagestyle{empty}

\makeatletter
\renewcommand{\@maketitle} {\begin{center} {\LARGE \textbf{ \textsc{\@title}} \par} \bigskip {\large \textbf{\textsc{\@author}} }\end{center} }
\renewcommand{\thispagestyle}[1]{}
\markright{\textbf{\textsc{AIF401 \textemdash Rencana Kerja Skripsi \textemdash Sem. Ganjil 2015/2016}}}

\onehalfspacing
 
\begin{document}

\title{\@judultopik}
\author{\nama \textendash \@npm} 

%tulis nama dan NPM anda di sini:
\newcommand{\nama}{Steven Sutana}
\newcommand{\@npm}{2012730046}
\newcommand{\@judultopik}{Porting PHP menjadi Play Framework (KIRI \textit{Front-End})} % Judul/topik anda
\newcommand{\jumpemb}{1} % Jumlah pembimbing, 1 atau 2
\newcommand{\tanggal}{26/08/2015}
\maketitle

\pagenumbering{arabic}

\section{Deskripsi}
KIRI Travel merupakan aplikasi yang membantu pengguna bepergian baik dalam kota maupun luar kota. Jika dalam kota, KIRI Travel akan menentukan angkutan kota yang tersedia di kota tersebut, jika luar kota, maka KIRI Travel menentukan travel yang tersedia ke kota yang akan dituju. KIRI Travel tersedia dalam berbagai kota, yaitu Bandung, Depok, Jakarta, Surabaya, dan Malang. KIRI Travel menyediakan berbagai rute alternatif yang dapat dipilih oleh pengguna. KIRI Travel juga dapat membimbing pengguna langkah demi langkah untuk mencapai lokasi tujuan. 


\section{Rumusan Masalah}
\begin{itemize}
	\item Bagaimana memahami dan menganalisa kode yang sudah ada?
	\item Bagaimana melakukan porting PHP menjadi Java (Play Framework) ?
\end{itemize}

\section{Tujuan}
\begin{itemize}
	\item Melakukan pengujian pada bahasa PHP tahap demi tahap.
	\item Menjadikan bahasa PHP menjadi Java yang lebih terstruktur dan umum.
\end{itemize}

\section{Deskripsi Perangkat Lunak}
Perangkat lunak akhir yang akan dibuat memiliki fitur minimal sebagai berikut:
\begin{itemize}
	\item Perangkat lunak dapat menampilkan kiri.travel secara terstruktur dengan menggunakan bahasa Java.
	\item Perangkat lunak dapat berfungsi sepenuhnya (\textit{Front-End})
\end{itemize}

\section{Rencana Kerja}

Rencana kerja untuk menyelesaikan skripsi ini:
\begin{itemize}
	\item Pada saat mengambil kuliah AIF401 Skripsi 1
	\begin{enumerate}
		\item Memahami dan melakukan analisa kode PHP yang sudah ada.
		\item Melakukan studi literatur tentang algoritma-algoritma yang berkaitan dengan pemrosesan data PHP menjadi Java.
		\item Mempelajari algoritma untuk melakukan porting PHP menjadi Java.
		\item Mempelajari fitur-fitur bahasa Java untuk membuat website \textit{Front-End}
		\item Merancang langkah-langkah untuk membuat website di Play Framework
	\end{enumerate}
	\item Pada saat mengambil kuliah AIF401 Skripsi 2
	\begin{enumerate}
		\item Merancang dan mengimplementasikan algoritma porting PHP menjadi Java
		\item Mengimplementasikan Play Framework
		\item Melakukan pengujian dan eksperimen
		\item Membuat dokumentasi skripsi
	\end{enumerate}
\end{itemize}

\section{Isi {\it Progress Report} Skripsi 1}
Isi dari {\it Progress Report} Skripsi 1 yang akan diselesaikan dan dilaporkan ke pembimbing paling lambat 2 minggu sebelum tenggat waktu yang ditetapkan koordinator adalah :
\begin{enumerate}
	\item Algoritma/langkah-langkah untuk membuat pembangkit otomatis data trajectory
	\item Hasil eksperimen penggunaan fitur-fitur Graphical User Interface pada bahasa Java
	\item Algoritma dan contoh perhitungan untuk kasus menghitung jarak dengan Frechet Distance
	\item \ldots (to be continued)
\end{enumerate}
Estimasi persentase penyelesaian skripsi sampai dengan {\it Progress Report} Skripsi 1 adalah : 99\%

\section{Pernyataan Khusus}
Berlatar belakang perihal kejujuran serta keterbasan jumlah dosen, saya menyatakan akan mematuhi aturan-aturan khusus berikut:
\begin{enumerate}
	\item Skripsi adalah hasil karya saya sendiri. Peran teman / orang lain adalah untuk membantu pemahaman, tetapi tidak dalam konten Skripsi.
	\item Saya menetapkan batasan yang jelas antara konten saya, dengan buatan orang lain (termasuk kode yang diambil dari {\it open source project})
	\item Pengambilan kedua hanya akan dilakukan hanya jika sudah memenuhi minimal 90\% dari target.
\end{enumerate}
Saya bersedia mematuhi peraturan di atas, dan bersedia menerima sanksi pembatalan pengambilan Skripsi dengan dosen pembimbing terkait jika terbukti melanggar. Peraturan ini berlaku pada Skripsi 1 dan 2.

\vspace{1.5cm}

\centering Bandung, \tanggal\\
\vspace{2cm} \nama \\ 
\vspace{1cm}

Menyetujui, \\
\ifdefstring{\jumpemb}{2}{
\vspace{1.5cm}
\begin{centering} Menyetujui,\\ \end{centering} \vspace{0.75cm}
\begin{minipage}[b]{0.45\linewidth}
% \centering Bandung, \makebox[0.5cm]{\hrulefill}/\makebox[0.5cm]{\hrulefill}/2013 \\
\vspace{2cm} Nama: \makebox[3cm]{\hrulefill}\\ Pembimbing Utama
\end{minipage} \hspace{0.5cm}
\begin{minipage}[b]{0.45\linewidth}
% \centering Bandung, \makebox[0.5cm]{\hrulefill}/\makebox[0.5cm]{\hrulefill}/2013\\
\vspace{2cm} Nama: \makebox[3cm]{\hrulefill}\\ Pembimbing Pendamping
\end{minipage}
\vspace{0.5cm}
}{
% \centering Bandung, \makebox[0.5cm]{\hrulefill}/\makebox[0.5cm]{\hrulefill}/2013\\
\vspace{2cm} Nama: \makebox[3cm]{\hrulefill}\\ Pembimbing Tunggal
}

\end{document}

